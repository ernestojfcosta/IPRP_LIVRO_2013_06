\documentclass[12pt,portuguese,a4paper]{article}
\usepackage[T1]{fontenc}
\usepackage[scaled]{luximono}
\usepackage{amsmath}
\usepackage{babel}
\usepackage{graphicx}
\usepackage{epstopdf}
\DeclareGraphicsRule{.tif}{png}{.png}{`convert #1 `dirname #1`/`basename #1 .tif`.png}
%\usepackage{picins} % problema com a package
\usepackage[applemac]{inputenc}
\usepackage{natbib}
\usepackage{color}
\usepackage{colortbl}
\usepackage[colorlinks,urlcolor=blue]{hyperref}
\usepackage{fancybox,calc}
\usepackage{alltt}
\usepackage[Lenny]{fncychap}
\usepackage{makeidx}
\usepackage{fancyvrb}
\usepackage{algorithm2e}
%% Customiza\c c�es

%%% PACKAGES
\usepackage{booktabs} % for much better looking tables
\usepackage{array} % for better arrays (eg matrices) in maths
\usepackage{paralist} % very flexible & customisable lists (eg. enumerate/itemize, etc.)
\usepackage{verbatim} % adds environment for commenting out blocks of text & for better verbatim
\usepackage{subfigure} % make it possible to include more than one captioned figure/table in a single float
% These packages are all incorporated in the memoir class to one degree or another...


%% END Article customise

%%% Para o c�digo
\usepackage{listings}
\lstloadlanguages{Python,C,Java,Prolog,Lisp}

\lstset{backgroundcolor=\color{yellow},language=Python,captionpos=b,showstringspaces=false,basicstyle=\ttfamily,commentstyle=\color{blue},tabsize=2,breaklines=true}

%% E mais umas coisas
\usepackage{pifont}
\usepackage{lettrine}
\usepackage{soul}
\usepackage{epigraph}

% See the ``Article customise'' template for come common customisations

\title{\Large{\textbf{Manipula\c c�o de imagens e Python}}}
\author{Ernesto Costa \\ Departamento de Engenharia Inform�tica \\ Universidade de Coimbra}
\date{\today} % delete this line to display the current date


\renewcommand\lstlistingname{Listagem}


%%% BEGIN DOCUMENT
\begin{document}




\maketitle


\begin{abstract}
Vivemos num mundo cada vez mais dominado pelas imagens. A manipula\c c�o das imagens permite-nos criar uma nova realidade e explorar novas combina\c c�es de formas e de cores. Com os computadores essa possibilidade de altera\c c�o e de jogo foi potenciada a uma escala nunca antes vista. Hoje, existem imensos programas que nos permitem exercitar a nossa imagina\c c�o art�stica, de que o \textbf{Photoshop} da Adobe � apenas um exemplo. Este pequeno texto pretende mostrar, de um modo elementar, como podemos n�s pr�prios usar e transformar imagens, por recurso � linguagem \textbf{Python} e ao m�dulo \textbf{cImage}.
\end{abstract}


\section{Computadores e Imagens}

\section{Python e Imagens}

\section{O m�dulo cImage}

\section{Exemplos}






\section{Conclus�o}



%\tableofcontents
\end{document}%% LaTeX - Article customise

