%%% cap08
%%% 

\chapter{M�dulos}

\begin{objectivos}

\item
\item

\end{objectivos}
\begin{table}[!htdp]
\caption{Fun��es}
	\begin{center}
	\begin{tabular}{ll}
		\rowcolor[gray]{0.7} Papel & Forma  \\ [0.5ex]
		Defini��o & \textbf{def} \texttt{toto(nome):} \textbf{print} \texttt{nome}\\
		Refer�ncia & \texttt{toto} \\
		Chamada & \texttt{toto('Ernesto')}\\ \hline
	\end{tabular}
	\end{center}
\label{tab:func}
\end{table}

\begin{table}[!htdp]
\caption{Fun��es: tipos de argumentos}
	\begin{center}
	\begin{tabular}{ll}
	\rowcolor[gray]{0.7} Sintaxe &  Interpreta��o  \\ [0.5ex]
		\texttt{func(valor)} &  Argumentos normais: posi��o \\
		\texttt{func(nome=valor)} &   Palavra chave: por nome \\ 
		\texttt{def func(nome)}&  argumento normal: por posi��o ou nome\\
		\texttt{def func(nome=valor)}& Por defeito. Pode ser alterado na chamada\\
		\texttt{def func(*nome)} &    Argumentos restantes por posi��o (num tuplo) \\
		\texttt{def func(**nome)} &Argumentos Restantes por palavra chave (num dicion�rio)\\ \hline
	\end{tabular}
	\end{center}
\label{tab:args}
\end{table}

\section*{Sum�rio}
\addcontentsline{toc}{section}{Sum�rio}

\section*{Leituras Adicionais}
\addcontentsline{toc}{section}{Leituras Adicionais}

\section*{Exerc�cios}
\addcontentsline{toc}{section}{Exerc�cios}


